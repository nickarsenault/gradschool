\documentclass[a4paper,12pt]{article} % This defines the style of your paper
\usepackage[top = 2.5cm, bottom = 2.5cm, left = 2.5cm, right = 2.5cm]{geometry} 
\usepackage[T1]{fontenc}
\usepackage[utf8]{inputenc}
\usepackage{newtxtext}
\usepackage[varvw]{newtxmath}
\let\openbox\undefined
\let\Bbbk\undefined
\usepackage{multirow} % Multirow is for tables with multiple rows within one cell.
\usepackage{booktabs} % For even nicer tables.
\usepackage{graphicx} 
\usepackage{setspace}
\setlength{\parindent}{0mm}
\usepackage{float}
\usepackage{fancyhdr}
\usepackage{amsmath,amsthm,verbatim,amssymb,amsfonts,amscd, graphicx}
\usepackage{physics}
\usepackage{float}
\usepackage{bigints}
\usepackage[mathscr]{eucal}
\usepackage{indentfirst}
\usepackage{graphicx}
\usepackage{xcolor}
\usepackage{transparent}
\usepackage{import}
\usepackage{pict2e}
\usepackage{epic}
\usepackage{epstopdf} 
\theoremstyle{plain}
\newtheorem{remark}{Remark}
\newtheorem{theorem}{Theorem}
\newtheorem{corollary}{Corollary}
\newtheorem{lemma}{Lemma}
\newtheorem{proposition}{Proposition}
\newtheorem*{surfacecor}{Corollary 1}
\newtheorem{conjecture}{Conjecture} 
\newtheorem{question}{Question} 
\theoremstyle{definition}
\usepackage{indentfirst}
\newtheorem{definition}{Definition}
\graphicspath{{/home/nickarsenault/Documents/HonorsThesis/Figures/}}
\usepackage{tcolorbox}
\tcbuselibrary{theorems}
\newtcbtheorem[number within=section]{mytheo}{Theorem}%
{colback=white!5,colframe=black!35!black,fonttitle=\bfseries}{th}
% Custom Macros
\newcommand{\C}{\mathbb{C}}
\newcommand{\B}{\mathcal{B}}
\newcommand{\N}{\mathbb{N}}
\newcommand{\Z}{\mathbb{Z}}
\newcommand{\R}{\mathbb{R}}
\newcommand{\Q}{\mathbb{Q}}
\newcommand{\open}{\mathcal{O}}
\newcommand{\cantor}{\mathcal{C}}
\newcommand{\nbr}{\mathcal{N}}
\newcommand{\cov}{\mathcal{U}}
\newcommand{\m}{m_*}
\newcommand{\D}{\Omega}
\newcommand{\bD}{{\partial \Omega}}
\newcommand{\lcm}{\text{lcm}}
\newcommand{\lnorm}{\left\|}
\newcommand{\rnorm}{\right\|}
\newcommand{\inv}{^{-1}}
\newcommand{\p}{^\prime}
\newcommand{\pp}{^{\prime \prime}}
\newcommand{\eps}{\varepsilon}
\newcommand{\nlim}{\lim_{n \to \infty}}
\newcommand{\suminf}{\sum_{n=1}^{\infty}}
\newcommand{\powsum}{\sum_{n=0}^{\infty}}
\newcommand{\G}{\Gamma}
\renewcommand{\c}{^{\mathsf{c}}}
\newcommand{\supp}{\text{supp}}
\newcommand{\finish}{\textcolor{red}{\textbf{Finish}}}
\newcommand{\half}{\textcolor{orange}{\textbf{Half}}}
\newcommand{\good}{\textcolor{green}{\textbf{Good}}}
\usepackage{xparse}
\ExplSyntaxOn
\NewDocumentCommand{\cycle}{ O{\;} m }
 {
  (
  \alec_cycle:nn { #1 } { #2 }
  )
 }

\seq_new:N \l_alec_cycle_seq
\cs_new_protected:Npn \alec_cycle:nn #1 #2
 {
  \seq_set_split:Nnn \l_alec_cycle_seq { , } { #2 }
  \seq_use:Nn \l_alec_cycle_seq { #1 }
 }
\ExplSyntaxOff

\usepackage{thmtools}

\declaretheoremstyle[
spaceabove=6pt, spacebelow=6pt,
headfont=\normalfont\bfseries,
notefont=\mdseries, notebraces={(}{)},
bodyfont=\normalfont,
postheadspace=1em,
numberwithin=section
]{exstyle}
\declaretheoremstyle[
spaceabove=6pt, spacebelow=6pt,
headfont=\normalfont\bfseries,
notefont=\mdseries, notebraces={(}{)},
bodyfont=\normalfont,
postheadspace=1em,
headpunct={},
qed=$\blacktriangleleft$,
numbered=no
]{solstyle}
\declaretheorem[style=exstyle]{example}
\declaretheorem[style=solstyle]{solution}
\pagestyle{fancy} % With this command we can customize the header style.
\fancyhf{} % This makes sure we do not have other information in our header or footer.
\lhead{\footnotesize Complex Analysis}% \lhead puts text in the top left corner. \footnotesize sets our font to a smaller size.
\rhead{\footnotesize Arsenault} %<---- Fill in your lastnames.
\cfoot{\footnotesize \thepage} 
\begin{document}
\thispagestyle{empty} % This command disables the header on the first page. 
\begin{tabular}{p{15.5cm}} % This is a simple tabular environment to align your text nicely 
{\large \bf Complex Analysis} \\
University of Kentucky \\ Spring 2022 \\
\hline % \hline produces horizontal lines.
\\
\end{tabular} % Our tabular environment ends here.

\vspace*{0.3cm} % Now we want to add some vertical space in between the line and our title.

\begin{center} % Everything within the center environment is centered.
  {\Large \bf Homework 4} % <---- Don't forget to put in the right number
	\vspace{2mm} \\
  \textbf{Nick Arsenault}
\end{center}  
\vspace{0.4cm}
\begin{enumerate}
\item[\textbf{\# 1.}]
  If $n = 2k$ for $k \in \Z$, then
  \begin{align*}
    \sin(\pi n) = \frac{e^{2 \pi i k} - e^{-2 \pi i k}}{2i} = 0
  \end{align*}
  since $e^{2 \pi i} = e^{-2 \pi i} = 1$.  Furthermore, if $n = 2k+1$ for $k \in \Z$, then
  \begin{align*}
    \sin(\pi n) = \frac{e^{i \pi (2k+1)} - e^{-i \pi(2k+1)}}{2i} = \frac{e^{i \pi}- e^{-i \pi}}{2i} = 0
  \end{align*} 
  since $e^{i \pi} = e^{-i \pi} = -1$.  Additionally, they are simple zeros because 
  \begin{align*}
    \frac{d}{dz}\sin(\pi z) = \pi \cos(\pi z) =\pi \frac{e^{i \pi z} + e^{-i \pi z}}{2}
  \end{align*}
  and $\cos(\pi n) \neq 0$ for $n \in \Z$.  This tells us that $n \in \Z$ are simple poles of $\frac{1}{\sin(\pi z)}$ and to calculate the residue we can use 
  \begin{align*}
    \Res_n \frac{1}{\sin(\pi z)} = \lim_{z \to n} \frac{(z-n)}{\sin(\pi z)} = \lim_{z \to n} \frac{1}{\pi \cos(\pi z)}
  \end{align*}
  by L'Hopital's rule.  If $n$ is even, then the limit equals $\frac{1}{\pi}$ and if $n$ is odd, the limit equals $-\frac{1}{\pi}$.
%%%%%%%%%%%%%%%%%%%%%%%%%%%%%%%%%%%%%%%%%%%%%%%%%%%%%%%%%%%%%%%%%%%%%%%%%%%%%%%%%%%%%%%%%%%%%%%%%%%%%%%%%%%%%%%%%%%%%%%%%%%%%%%%%%%%%%%%%%%%%%%%%%%%%%%%%%%%%%%%%%%%%%%%%%%%%%%%%%
%%%%%%%%%%%%%%%%%%%%%%%%%%%%%%%%%%%%%%%%%%%%%%%%%%%%%%%%%%%%%%%%%%%%%%%%%%%%%%%%%%%%%%%%%%%%%%%%%%%%%%%%%%%%%%%%%%%%%%%%%%%%%%%%%%%%%%%%%%%%%%%%%%%%%%%%%%%%%%%%%%%%%%%%%%%%%%%%%%
%%%%%%%%%%%%%%%%%%%%%%%%%%%%%%%%%%%%%%%%%%%%%%%%%%%%%%%%%%%%%%%%%%%%%%%%%%%%%%%%%%%%%%%%%%%%%%%%%%%%%%%%%%%%%%%%%%%%%%%%%%%%%%%%%%%%%%%%%%%%%%%%%%%%%%%%%%%%%%%%%%%%%%%%%%%%%%%%%%
%%%%%%%%%%%%%%%%%%%%%%%%%%%%%%%%%%%%%%%%%%%%%%%%%%%%%%%%%%%%%%%%%%%%%%%%%%%%%%%%%%%%%%%%%%%%%%%%%%%%%%%%%%%%%%%%%%%%%%%%%%%%%%%%%%%%%%%%%%%%%%%%%%%%%%%%%%%%%%%%%%%%%%%%%%%%%%%%%%
\item[\textbf{\# 2.}]
  Let $\Gamma$ be the semi-circle contour composed of the line segment from  $-R$ to $R$, and $\Gamma_R$ the arc of radius $R$.  Then
  \begin{align*}
    \int_{\Gamma} \frac{1}{1 + z^{4}} dz = \int_{-R}^{R} \frac{1}{1 + z^{4}} dz + \int_{\Gamma_R} \frac{1}{1+z^{4}} dz.
  \end{align*}
  Furthermore, 
  \begin{align*}
    \abs{\frac{1}{1+z^{4}}} \le  \abs{\frac{1}{R^{4} - 1}}
  \end{align*}
  by the reverse triangle inequality.  Hence
  \begin{align*}
    \abs{\int_{\Gamma_R} \frac{1}{1+z^{4}}dz} \le \frac{\pi}{R^3} \to 0 
  \end{align*}
  as $R \to \infty$. The integrand has a simple poles inside $\Gamma$ at $e^{i \frac{\pi}{4}}$ and $e^{i\frac{3\pi}{4}}$.  Knowing this, we can calculate the residue by
  \begin{align*}
    \Res_{e^{i \frac{\pi}{4}}}\frac{1}{1+z^{4}} &= \lim_{z \to e^{i \frac{\pi}{4}}} \frac{z - e^{i \frac{\pi}{4}}}{1 + z^{4}}\\
                                                &= \frac{1}{4e^{i \frac{3\pi}{4}}} \\
                                                &= \frac{1}{4} \left(\frac{-1- i}{\sqrt{2}} \right) 
  \end{align*}
  and
  \begin{align*}
    \Res_{e^{i \frac{3\pi}{4}}}\frac{1}{1+z^{4}} &= \lim_{z \to e^{i \frac{3\pi}{4}}} \frac{z - e^{i \frac{\pi}{4}}}{1 + z^{4}}\\
                                                 &= \frac{1}{4 e^{i \frac{\pi}{4}}} \\
                                                 &= \frac{1}{4}\left(\frac{1 -i}{\sqrt{2}}\right).
  \end{align*}
  Therefore, sending $R \to \infty$ and using Cauchy's Residue theorem 
  \begin{align*}
    \int_{-\infty}^{\infty} \frac{1}{1+x^{4}}dx = 2 \pi i \left(\frac{-i}{2\sqrt{2}}\right) = \frac{\pi}{\sqrt{2}}. 
  \end{align*}
%%%%%%%%%%%%%%%%%%%%%%%%%%%%%%%%%%%%%%%%%%%%%%%%%%%%%%%%%%%%%%%%%%%%%%%%%%%%%%%%%%%%%%%%%%%%%%%%%%%%%%%%%%%%%%%%%%%%%%%%%%%%%%%%%%%%%%%%%%%%%%%%%%%%%%%%%%%%%%%%%%%%%%%%%%%%%%%%%%
%%%%%%%%%%%%%%%%%%%%%%%%%%%%%%%%%%%%%%%%%%%%%%%%%%%%%%%%%%%%%%%%%%%%%%%%%%%%%%%%%%%%%%%%%%%%%%%%%%%%%%%%%%%%%%%%%%%%%%%%%%%%%%%%%%%%%%%%%%%%%%%%%%%%%%%%%%%%%%%%%%%%%%%%%%%%%%%%%%
%%%%%%%%%%%%%%%%%%%%%%%%%%%%%%%%%%%%%%%%%%%%%%%%%%%%%%%%%%%%%%%%%%%%%%%%%%%%%%%%%%%%%%%%%%%%%%%%%%%%%%%%%%%%%%%%%%%%%%%%%%%%%%%%%%%%%%%%%%%%%%%%%%%%%%%%%%%%%%%%%%%%%%%%%%%%%%%%%%
%%%%%%%%%%%%%%%%%%%%%%%%%%%%%%%%%%%%%%%%%%%%%%%%%%%%%%%%%%%%%%%%%%%%%%%%%%%%%%%%%%%%%%%%%%%%%%%%%%%%%%%%%%%%%%%%%%%%%%%%%%%%%%%%%%%%%%%%%%%%%%%%%%%%%%%%%%%%%%%%%%%%%%%%%%%%%%%%%%
\item[\textbf{\# 3.}]
  Notice that
  \begin{align*}
    \int_{-\infty}^{\infty} \frac{\cos(x)}{x^2 + a^2} dx = \Re\int_S \frac{e^{iz}}{z^{2} + a^2} dz 
  \end{align*}
  where $S$ is the semi-circle contour.  Additionally, if $\Gamma_R$ is the arc of radius $R$, then
   \begin{align*}
   \abs{\int_{\Gamma_R} \frac{e^{iz}}{z^2 + a^2} dz} \le \pi R \cdot \frac{1}{R^2} \to 0
  \end{align*}
  as $R \to \infty$.  Thus,
  \begin{align*}
    \int_{-\infty}^{\infty} \frac{e^{ix}}{x^2 + a^2} dx = 2 \pi i \Res_{ia} \frac{e^{iz}}{z^2 + a^2} = 2 \pi i \frac{e^{-a}}{2ai} = \frac{\pi}{a} e^{-a}.
  \end{align*}
  Consequently,
  \begin{align*}
    \int_{-\infty}^{\infty} \frac{\cos(x)}{x^2+a^2} dx = \frac{\pi}{a} e^{-a}.
  \end{align*}
%%%%%%%%%%%%%%%%%%%%%%%%%%%%%%%%%%%%%%%%%%%%%%%%%%%%%%%%%%%%%%%%%%%%%%%%%%%%%%%%%%%%%%%%%%%%%%%%%%%%%%%%%%%%%%%%%%%%%%%%%%%%%%%%%%%%%%%%%%%%%%%%%%%%%%%%%%%%%%%%%%%%%%%%%%%%%%%%%%
%%%%%%%%%%%%%%%%%%%%%%%%%%%%%%%%%%%%%%%%%%%%%%%%%%%%%%%%%%%%%%%%%%%%%%%%%%%%%%%%%%%%%%%%%%%%%%%%%%%%%%%%%%%%%%%%%%%%%%%%%%%%%%%%%%%%%%%%%%%%%%%%%%%%%%%%%%%%%%%%%%%%%%%%%%%%%%%%%%
%%%%%%%%%%%%%%%%%%%%%%%%%%%%%%%%%%%%%%%%%%%%%%%%%%%%%%%%%%%%%%%%%%%%%%%%%%%%%%%%%%%%%%%%%%%%%%%%%%%%%%%%%%%%%%%%%%%%%%%%%%%%%%%%%%%%%%%%%%%%%%%%%%%%%%%%%%%%%%%%%%%%%%%%%%%%%%%%%%
%%%%%%%%%%%%%%%%%%%%%%%%%%%%%%%%%%%%%%%%%%%%%%%%%%%%%%%%%%%%%%%%%%%%%%%%%%%%%%%%%%%%%%%%%%%%%%%%%%%%%%%%%%%%%%%%%%%%%%%%%%%%%%%%%%%%%%%%%%%%%%%%%%%%%%%%%%%%%%%%%%%%%%%%%%%%%%%%%%
\item[\textbf{\# 8.}]
  We have that 
  \begin{align*}
  \cos(\theta) = \frac{1}{2}\left(e^{i\theta} + e^{-i\theta} \right) = \frac{1}{2}\left( z + \frac{1}{z} \right)
  \end{align*}
  where $z = e^{i\theta}$ for $0 \le  \theta \le 2\pi$.  Then
  \begin{align*}
    \int_{0}^{2 \pi } \frac{d\theta}{a + b \cos(\theta)} = -\frac{2i}{b} \int_{C} \frac{1}{z^2 +\frac{2a}{b}z + 1} dz.
  \end{align*}
  The integrand has poles at 
  \begin{align*}
    \alpha = \frac{-a -\sqrt{a^2 + b^2}}{b} \quad \text{and} \quad \beta = \frac{-a -\sqrt{a^2-b^2}  }{b}.
  \end{align*}
However, only $\alpha$ lies in $C$.  Therefore, by Cauchy's Residue theorem
\begin{align*}
  \int_{0}^{2\pi} \frac{d \theta}{a + b \cos(\theta)} = -\frac{2i}{b} \Res_{\alpha} \frac{1}{z^2 + \frac{2a}{b}z + 1} = \frac{2\pi}{\sqrt{a^2-b^2} }.
\end{align*}
  
%%%%%%%%%%%%%%%%%%%%%%%%%%%%%%%%%%%%%%%%%%%%%%%%%%%%%%%%%%%%%%%%%%%%%%%%%%%%%%%%%%%%%%%%%%%%%%%%%%%%%%%%%%%%%%%%%%%%%%%%%%%%%%%%%%%%%%%%%%%%%%%%%%%%%%%%%%%%%%%%%%%%%%%%%%%%%%%%%%
%%%%%%%%%%%%%%%%%%%%%%%%%%%%%%%%%%%%%%%%%%%%%%%%%%%%%%%%%%%%%%%%%%%%%%%%%%%%%%%%%%%%%%%%%%%%%%%%%%%%%%%%%%%%%%%%%%%%%%%%%%%%%%%%%%%%%%%%%%%%%%%%%%%%%%%%%%%%%%%%%%%%%%%%%%%%%%%%%%
%%%%%%%%%%%%%%%%%%%%%%%%%%%%%%%%%%%%%%%%%%%%%%%%%%%%%%%%%%%%%%%%%%%%%%%%%%%%%%%%%%%%%%%%%%%%%%%%%%%%%%%%%%%%%%%%%%%%%%%%%%%%%%%%%%%%%%%%%%%%%%%%%%%%%%%%%%%%%%%%%%%%%%%%%%%%%%%%%%
%%%%%%%%%%%%%%%%%%%%%%%%%%%%%%%%%%%%%%%%%%%%%%%%%%%%%%%%%%%%%%%%%%%%%%%%%%%%%%%%%%%%%%%%%%%%%%%%%%%%%%%%%%%%%%%%%%%%%%%%%%%%%%%%%%%%%%%%%%%%%%%%%%%%%%%%%%%%%%%%%%%%%%%%%%%%%%%%%%
 \end{enumerate} 
\end{document}


