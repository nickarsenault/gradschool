\documentclass[a4paper,12pt]{article} % This defines the style of your paper
\usepackage[top = 2.5cm, bottom = 2.5cm, left = 2.5cm, right = 2.5cm]{geometry} 
\usepackage[T1]{fontenc}
\usepackage[utf8]{inputenc}
\usepackage{newtxtext}
\usepackage[varvw]{newtxmath}
\let\openbox\undefined
\let\Bbbk\undefined
\usepackage{multirow} % Multirow is for tables with multiple rows within one cell.
\usepackage{booktabs} % For even nicer tables.
\usepackage{graphicx} 
\usepackage{setspace}
\setlength{\parindent}{0mm}
\usepackage{float}
\usepackage{fancyhdr}
\usepackage{amsmath,amsthm,verbatim,amssymb,amsfonts,amscd, graphicx}
\usepackage{physics}
\usepackage{float}
\usepackage{bigints}
\usepackage[mathscr]{eucal}
\usepackage{indentfirst}
\usepackage{graphicx}
\usepackage{xcolor}
\usepackage{transparent}
\usepackage{import}
\usepackage{pict2e}
\usepackage{epic}
\usepackage{epstopdf} 
\theoremstyle{plain}
\newtheorem{remark}{Remark}
\newtheorem{theorem}{Theorem}
\newtheorem{corollary}{Corollary}
\newtheorem{lemma}{Lemma}
\newtheorem{proposition}{Proposition}
\newtheorem*{surfacecor}{Corollary 1}
\newtheorem{conjecture}{Conjecture} 
\newtheorem{question}{Question} 
\theoremstyle{definition}
\usepackage{indentfirst}
\newtheorem{definition}{Definition}
\graphicspath{{/home/nickarsenault/Documents/HonorsThesis/Figures/}}
\usepackage{tcolorbox}
\tcbuselibrary{theorems}
\newtcbtheorem[number within=section]{mytheo}{Theorem}%
{colback=white!5,colframe=black!35!black,fonttitle=\bfseries}{th}
% Custom Macros
\newcommand{\C}{\mathbb{C}}
\newcommand{\B}{\mathcal{B}}
\newcommand{\N}{\mathbb{N}}
\newcommand{\Z}{\mathbb{Z}}
\newcommand{\R}{\mathbb{R}}
\newcommand{\Q}{\mathbb{Q}}
\newcommand{\open}{\mathcal{O}}
\newcommand{\cantor}{\mathcal{C}}
\newcommand{\nbr}{\mathcal{N}}
\newcommand{\cov}{\mathcal{U}}
\newcommand{\m}{m_*}
\newcommand{\D}{\Omega}
\newcommand{\bD}{{\partial \Omega}}
\newcommand{\lcm}{\text{lcm}}
\newcommand{\lnorm}{\left\|}
\newcommand{\rnorm}{\right\|}
\newcommand{\inv}{^{-1}}
\newcommand{\p}{^\prime}
\newcommand{\pp}{^{\prime \prime}}
\newcommand{\eps}{\varepsilon}
\newcommand{\nlim}{\lim_{n \to \infty}}
\newcommand{\suminf}{\sum_{n=1}^{\infty}}
\newcommand{\powsum}{\sum_{n=0}^{\infty}}
\newcommand{\G}{\Gamma}
\renewcommand{\c}{^{\mathsf{c}}}
\newcommand{\supp}{\text{supp}}
\newcommand{\finish}{\textcolor{red}{\textbf{Finish}}}
\newcommand{\half}{\textcolor{orange}{\textbf{Half}}}
\newcommand{\good}{\textcolor{green}{\textbf{Good}}}
\usepackage{xparse}
\ExplSyntaxOn
\NewDocumentCommand{\cycle}{ O{\;} m }
 {
  (
  \alec_cycle:nn { #1 } { #2 }
  )
 }

\seq_new:N \l_alec_cycle_seq
\cs_new_protected:Npn \alec_cycle:nn #1 #2
 {
  \seq_set_split:Nnn \l_alec_cycle_seq { , } { #2 }
  \seq_use:Nn \l_alec_cycle_seq { #1 }
 }
\ExplSyntaxOff

\usepackage{thmtools}

\declaretheoremstyle[
spaceabove=6pt, spacebelow=6pt,
headfont=\normalfont\bfseries,
notefont=\mdseries, notebraces={(}{)},
bodyfont=\normalfont,
postheadspace=1em,
numberwithin=section
]{exstyle}
\declaretheoremstyle[
spaceabove=6pt, spacebelow=6pt,
headfont=\normalfont\bfseries,
notefont=\mdseries, notebraces={(}{)},
bodyfont=\normalfont,
postheadspace=1em,
headpunct={},
qed=$\blacktriangleleft$,
numbered=no
]{solstyle}
\declaretheorem[style=exstyle]{example}
\declaretheorem[style=solstyle]{solution}
\pagestyle{fancy} % With this command we can customize the header style.
\fancyhf{} % This makes sure we do not have other information in our header or footer.
\lhead{\footnotesize Complex Analysis}% \lhead puts text in the top left corner. \footnotesize sets our font to a smaller size.
\rhead{\footnotesize Arsenault} %<---- Fill in your lastnames.
\cfoot{\footnotesize \thepage} 
\begin{document}
\thispagestyle{empty} % This command disables the header on the first page. 
\begin{tabular}{p{15.5cm}} % This is a simple tabular environment to align your text nicely 
{\large \bf Complex Analysis} \\
University of Kentucky \\ Spring 2022 \\
\hline % \hline produces horizontal lines.
\\
\end{tabular} % Our tabular environment ends here.

\vspace*{0.3cm} % Now we want to add some vertical space in between the line and our title.

\begin{center} % Everything within the center environment is centered.
  {\Large \bf Homework 1} % <---- Don't forget to put in the right number
	\vspace{2mm} \\
  \textbf{Nick Arsenault}
\end{center}  
\vspace{0.4cm}
 \begin{enumerate}
   \item[\textbf{\# 1.}] 
     \begin{itemize}
       \item[(a.)]  This consists of all the $z \in \C$ that are equidistant from $z_1$ and $z_2$.
       \item[(b.)]  If $z \neq 0$, then we have $1 = z \overline{z} = \abs{z}^2$.  Hence, the set is $\{z \in \C, z \neq 0 : \abs{z} = 1\}$. 
       \item[(c.)]  The region is a vertical line through $x = 3$. 
       \item[(d.)]  The region is the complex plane to the right of $c$.  If we have $\Re z > c$, then we exclude the vertical line through  $x = c$.  If we have  $\Re z \ge c$, then the vertical line through $x =c$.
     \end{itemize}
   \item[\textbf{\# 2.}] 
     We have
     \begin{align*}
       \frac{1}{2}[(z,w) + (w,z)] &= \frac{1}{2}[z\overline{w} + w \overline{z}] \\
                                  &= \frac{1}{2}[(x_1x_2 + y_1 y_2 - iy_1 x_2 - i y_2 x_1) + (x_1 x_2 + y_1 y_2 + i y_2 x_1 - i y_1 x_2)] \\
                                  &= x_1 x_2 + y_1 y_2 \\
                                  &= \langle z, w\rangle.
     \end{align*}
     Moreover, from this calculation, we see that $\Re(z,w) = x_1x_2 + y_1 y_2 = \langle z, w\rangle$.
   \item[\textbf{\# 7.}] 
     \begin{itemize}
       \item[(a.)]  Let $w,z \in \C$ with the form $w = x_1 + i y_1$, $z = x_2 + i y_2$. Notice that 
         \begin{align*}
           \abs{w-z} &= \abs{(x_1 + i y_1) - (x_2 + i y_2)} \\
                                               &= \abs{(x_1 + i (y_1 - y_2)) - x_2} \\
                                               &= \abs{\tilde{w} - r}
         \end{align*}
         Therefore, we may assume that $z \in \R$, in which case  
         \begin{align*}
           (r - w)(r - \overline{w}) &<  (1-rw)(1 - r\overline{w}) \\
           r^2 - rw -r\overline{w} + w \overline{w} &< 1 - rw -r \overline{w} + r^2 w \overline{w} \\
            r^2 + w \overline{w} &< 1 + r^2 w \overline{w} \\
            r^2 - r^2 w \overline{w} &< 1 - w \overline{w} \\
            r^2(1 - w \overline{w}) &< 1 - w \overline{w}
         \end{align*}
         which holds for $\abs{r} < 1$ and $\abs{w} < 1$.  Therefore, we get that
\begin{align*}
  \abs{\frac{w-z}{1 - \overline{w}z}} < 1.
\end{align*}
         Furthermore, if $\abs{r} = 1$ and $\abs{w} = 1$, then
         \begin{align*}
           (r-w)(r - \overline{w}) &= r^2 - r \overline{w} - rw + w \overline{w} \\
                                   &= 1 - r \overline{w} - rw + r^2 \overline{w} \\
                                   &= (1- rw)(1 - r\overline{w}).
         \end{align*}
Thus, 
\begin{align*}
  \abs{\frac{w-z}{1 - \overline{w} z}} = 1.
\end{align*}
\item[(b.)]  
  \begin{itemize}
    \item[(\textit{i.})]
      For $w \in \mathbb{D}$ fixed and $z \in \mathbb{D}$, we have
      \begin{align*}
        \abs{F(z)} = \abs{\frac{w-z}{1-\overline{w}z}} < 1
      \end{align*}
      by part (a.).  Thus, $F$ maps $\mathbb{D}$ to itself.  Secondly,
      \begin{align*}
        \pdv{F}{\overline{z}} = \pdv{\overline{z}} \frac{w-z}{1 - \overline{w}z} = 0.
      \end{align*}
     Hence, $F$ is holomorphic. 
    \item[(\textit{ii.})]  First,
      \begin{align*}
        F(0) = \frac{w}{1 - \overline{w}\cdot 0} = w
      \end{align*}
      and secondly,
      \begin{align*}
        F(w) = \frac{w-w}{1 - \overline{w}w} = 0.
      \end{align*}
      We see that $F$ swaps the elements $0$ and $w$.
    \item[(\textit{iii.})]  If $\abs{z} = 1$, then by (a.), we have
      \begin{align*}
        \abs{F(z)} = \abs{\frac{w-z}{1-\overline{w}z}} = 1.
      \end{align*}
    \item[(\textit{iv.})]  Lastly, calculating $F \circ F$, gives
      \begin{align*}
        F(F(z)) &= \frac{w - \frac{w-z}{1-\overline{w}z}}{1 - \overline{w} \frac{w-z}{1-\overline{w}z}} \\
         &= \frac{\frac{w - w \overline{w}z}{1-\overline{w}z} - \frac{w-z}{1-\overline{w}z}}{\frac{1-\overline{w}z}{1-\overline{w}z} - \frac{\overline{w}w-\overline{w}z}{1-\overline{w}z}} \\
         &= \frac{\frac{- w \overline{w}z + z}{1-\overline{w}z}}{\frac{1-\overline{w}w}{1-\overline{w}z}} \\
        &=  z \\
      \end{align*}
      Therefore, $F$ is a bijection.
  \end{itemize}
     \end{itemize}
   \item[\textbf{\# 13.}] 
   \item[\textbf{\# 20.}] 
   \item[\textbf{\# 23.}] 
 \end{enumerate} 
\end{document}


