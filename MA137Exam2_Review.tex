\documentclass[a4paper,12pt]{article} % This defines the style of your paper
\usepackage[top = 2.5cm, bottom = 2.5cm, left = 2.5cm, right = 2.5cm]{geometry} 

\usepackage[T1]{fontenc}
\usepackage[utf8]{inputenc}
\usepackage{newtxtext}
\usepackage[varvw]{newtxmath}
\let\openbox\undefined
\let\Bbbk\undefined
\usepackage{multirow} % Multirow is for tables with multiple rows within one cell.
\usepackage{booktabs} % For even nicer tables.
\usepackage{graphicx} 
\usepackage{setspace}
\setlength{\parindent}{0mm}
\usepackage{float}
\usepackage{fancyhdr}
%%%%%%%%%%%%%%%%%%%%Nick's Additional Packages%%%%%%%%%%%%%%%%%%%%
\usepackage{amsmath,amsthm,verbatim,amssymb,amsfonts,amscd, graphicx}
\usepackage{graphicx}
\usepackage{physics}
\usepackage{float}
\usepackage{bigints}
\usepackage[mathscr]{eucal}
\usepackage{indentfirst}
\usepackage{graphicx}
\usepackage{graphics}
\usepackage{pict2e}
\usepackage{epic}
\usepackage{hyperref}
\usepackage{epstopdf} 
\theoremstyle{plain}
\newtheorem{remark}{Remark}
\newtheorem{theorem}{Theorem}
\newtheorem{corollary}{Corollary}
\newtheorem{lemma}{Lemma}
\newtheorem{proposition}{Proposition}
\newtheorem*{surfacecor}{Corollary 1}
\newtheorem{conjecture}{Conjecture} 
\newtheorem{question}{Question} 
\theoremstyle{definition}
\usepackage{indentfirst}
\newtheorem{definition}{Definition}
\graphicspath{{/home/nickarsenault/Documents/HonorsThesis/Figures/}}
\usepackage{tcolorbox}
\tcbuselibrary{theorems}
\newtcbtheorem[number within=section]{mytheo}{Theorem}%
{colback=white!5,colframe=black!35!black,fonttitle=\bfseries}{th}


% Custom Macros
\newcommand{\C}{\mathbb{C}}
\newcommand{\B}{\mathcal{B}}
\newcommand{\N}{\mathbb{N}}
\newcommand{\Z}{\mathbb{Z}}
\newcommand{\R}{\mathbb{R}}
\newcommand{\Q}{\mathbb{Q}}
\newcommand{\open}{\mathcal{O}}
\newcommand{\cantor}{\mathcal{C}}
\newcommand{\nbr}{\mathcal{N}}
\newcommand{\cov}{\mathcal{U}}
\newcommand{\m}{m_*}
\newcommand{\D}{\Omega}
\newcommand{\bD}{{\partial \Omega}}
\newcommand{\lcm}{\text{lcm}}
\newcommand{\lnorm}{\left\|}
\newcommand{\rnorm}{\right\|}
\newcommand{\inv}{^{-1}}
\newcommand{\p}{^\prime}
\newcommand{\pp}{^{\prime \prime}}
\newcommand{\eps}{\varepsilon}
\newcommand{\nlim}{\lim_{n \to \infty}}
\newcommand{\suminf}{\sum_{n=1}^{\infty}}
\newcommand{\powsum}{\sum_{n=0}^{\infty}}
\newcommand{\G}{\Gamma}
\newcommand{\hil}{\mathcal{H}}
\renewcommand{\c}{^{\mathsf{c}}}
\newcommand{\supp}{\text{supp}}
\newcommand{\finish}{\textcolor{red}{\textbf{Finish}}}
\newcommand{\half}{\textcolor{orange}{\textbf{Half}}}
\newcommand{\good}{\textcolor{green}{\textbf{Good}}}
\usepackage{xparse}

\ExplSyntaxOn
\NewDocumentCommand{\cycle}{ O{\;} m }
 {
  (
  \alec_cycle:nn { #1 } { #2 }
  )
 }

\seq_new:N \l_alec_cycle_seq
\cs_new_protected:Npn \alec_cycle:nn #1 #2
 {
  \seq_set_split:Nnn \l_alec_cycle_seq { , } { #2 }
  \seq_use:Nn \l_alec_cycle_seq { #1 }
 }
\ExplSyntaxOff
\DeclareMathOperator*{\esssup}{ess\,sup}
\DeclareMathOperator*{\essinf}{ess\,inf}
\usepackage{thmtools}

\declaretheoremstyle[
spaceabove=6pt, spacebelow=6pt,
headfont=\normalfont\bfseries,
notefont=\mdseries, notebraces={(}{)},
bodyfont=\normalfont,
postheadspace=1em,
numberwithin=section
]{exstyle}
\declaretheoremstyle[
spaceabove=6pt, spacebelow=6pt,
headfont=\normalfont\bfseries,
notefont=\mdseries, notebraces={(}{)},
bodyfont=\normalfont,
postheadspace=1em,
headpunct={},
qed=$\blacktriangleleft$,
numbered=no
]{solstyle}
\declaretheorem[style=exstyle]{example}
\declaretheorem[style=solstyle]{solution}
%%%%%%%%%%%%%%%%%%%%%%%%%%%%%%%%%%%%%%%%%%%%%%%%
% 3. Header (and Footer)
%%%%%%%%%%%%%%%%%%%%%%%%%%%%%%%%%%%%%%%%%%%%%%%%

% To make our document nice we want a header and number the pages in the footer.

\pagestyle{fancy} % With this command we can customize the header style.

\fancyhf{} % This makes sure we do not have other information in our header or footer.

\begin{document}
\thispagestyle{empty} % This command disables the header on the first page. 

\begin{tabular}{p{15.5cm}} % This is a simple tabular environment to align your text nicely 
{\large \bf MA137} \\
University of Kentucky \\ Fall 2021 \\
\hline % \hline produces horizontal lines.
\\
\end{tabular} % Our tabular environment ends here.

\vspace*{0.3cm} % Now we want to add some vertical space in between the line and our title.

\begin{center} % Everything within the center environment is centered.
	{\Large \bf Exam 2 Review} % <---- Don't forget to put in the right number
	\vspace{2mm}
	
        % YOUR NAMES GO HERE
\end{center}  

\vspace{0.4cm}

\begin{enumerate}

\item[\underline{\textbf{\#1}}] \textit{Use the chain rule to find the derivative of the following functions:
  \begin{itemize}
    \item[(\text{a.})] $f(N) = (1+3N^2)^3$
  \item[(\text{b.})] $g(s) = \sqrt[4]{5s^3-3}   $
  \end{itemize}}
\newpage
\item[\underline{\textbf{\#2}}] \textit{\begin{itemize}
    \item[(\text{a.})] The following limit represents the derivative $f\p(x_0)$ of a function $f$ at a point $x_0$ 
      \begin{align*}
        \lim_{h\to 0}\frac{4(2+h)^3-32}{h}
      \end{align*}
      Find $f$ and $x_0$. What is the value of the limit?
    \item[(\text{b.})] Let $g(x) = 5x^3-2x +\frac{2}{x} + \pi^2$. Compute $g\p(x)$.
  \end{itemize}}
 \newpage
\item[\underline{\textbf{\#3}}] \textit{Explain how to use Intermediate Value Theorem to conculde that the equation
  \begin{align*}
   \sqrt{x^2+2} = 2
  \end{align*}
has a solution in the interval $[1,2]$.}
\newpage
\item[\underline{\textbf{\#4}}] \textit{Suppose a function $y = y(x)$ is implicitly defined by the equation
  \begin{align*}
    y^3x^{4} -10x +y = -3
  \end{align*}
what is $\frac{dy}{dx}$ at the point $(2,1)$?}
\newpage
\item[\underline{\textbf{\#5}}] \textit{Suppose 
  \begin{align*}
    F(x) = \frac{1-x^2}{h(x)} 
  \end{align*}
and $h(2)=-1$ and $h\p(2)=-2$. Find  $F\p(2)$.}
\newpage
\item[\underline{\textbf{\#6}}] \textit{Assume that $f(x)$ is everywhere continuous and it is given to you that
  \begin{align*}
    \lim_{x \to 7} \frac{f(x)+9}{x-7}=10.
\end{align*}
It follows that $y = \text{?}$ is the equation of the tangent line at the point $(?,?)$.}
\newpage
\item[\underline{\textbf{\#7}}] \textit{Suppose that $-8x -22\le f(x) \le x^2 -2x -13$ for all $x$. Use this information and the Sandwich Theorem to compute 
  \begin{align*}
    \lim_{x\to -3} f(x).
  \end{align*}}
\newpage
\item[\underline{\textbf{\#8}}] \textit{Compute
  \begin{align*}
    \lim_{h\to 0} \frac{(h+4)^2 -16}{h}.
  \end{align*}}
\newpage
\item[\underline{\textbf{\#9}}] \textit{ Consider the function
  \begin{align*}
    f(x) = \begin{cases}
        c + 2x & x < 4 \\
        \frac{96}{x + c} & x \ge 4
    \end{cases}
  \end{align*}
Find all the values of $c$ such that $f(x)$ is continuous at $x = 4$.}
\newpage
\item[\underline{\textbf{\#10}}] \textit{Find the value of the limit 
  \begin{align*}
  \lim_{x \to 0} \frac{\tan(2x)}{3x}.
  \end{align*}}
\newpage
\item[\underline{\textbf{\#11}}] \textit{Suppose $y = 3x +2$ is the equation of the tangent line to a function $f$ at the point $(1,f(1))$. Find $f(1)$ and $f\p(1)$.}
\newpage

\item[\underline{\textbf{\#12}}] \textit{After a spill, a circular oil slick grows on the surface of a lake.  If the radius of the oil slick is increasing at a rate of $\frac{1}{2}$ foot per minute, how fast is the area of the oil slick increasing when its radus is $4$ feet?}
\newpage
\end{enumerate}

\end{document}
